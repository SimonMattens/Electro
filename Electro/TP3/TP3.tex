\documentclass{report}
\usepackage[pdftex]{graphicx}
\usepackage{sidecap}
\usepackage{fancyhdr}
\usepackage{lscape}
\usepackage[absolute]{textpos}
\usepackage{amssymb}
\usepackage{french}

\title{Rapport : \\ \'Etude des circuits RLC}
\author{Groupe : \\ Mattens Simon, Dom Eduardo \\ BAB2 Info}

\date{\today}


\begin{document}
\maketitle

\section*{1. Introduction}

\hspace*{0,5cm} Le but de l'exp\'erience \'etait l'\'etude de circuits aliment\'es en tension alternative et comprenant des associations de r\'esistances; condensateurs et bobines.\\
Celle-ci \'etait divis\'ee en deux manipulations principales:
\begin{itemize}
\item Etude d'un circuit RLC s\'erie (observation de la r\'esonance)
\item Etude d'un circuit RL \& C en parall\`ele (observation de l'anti-r\'esonance)
\end{itemize}

\section*{2. R\'esum\'e th\'eorie}
\subsection*{2.1 V\'erification th\'eorique}
Avant de commencer la manipulation, une v\'erification \'etait demand\'ee: \\

\textbf{\'Enonc\'e:} V\'erifier, par l'analyse dimensionnelle, que l'unit\'e de $\omega$L et de \\ 
\hspace*{2,07cm}$1/(\omega$C) est l'Ohm $[\Omega]$ \\ 

\textbf{R\'eponse:} \\

\textbf{$\bullet\ \omega L$:} \\
$$[\omega L] = \frac{1rad\cdot 1H}{1s} = \frac{1rad\cdot 1V\cdot 1s}{1A\cdot 1s\ }\textit{car\ } 1H = \frac{1V\cdot 1s}{1A}$$ 
$$= \frac{1rad\cdot 1V}{1A} = \frac{1V}{1A}$$ \begin{center}
\textit{car les angles sont des grandeurs sans dimensions} 
$$ = 1\Omega,\ CQFD $$
\end{center}

\textbf{$\bullet\ 1/(\omega$C):} \\
$$[\frac{1}{\omega C}] = \frac{1s}{1rad\cdot 1F} = \frac{1V\cdot 1s}{1rad\cdot 1C}\ \textit{car\ } 1F = \frac{1C}{1V}$$ 
$$ = \frac{1V\cdot 1s}{1rad\cdot 1A\cdot 1s}\ \textit{car\ } 1C = 1A\cdot 1s $$
$$= \frac{1V}{1rad\cdot 1A} = \frac{1V}{1A}$$ \begin{center}
\textit{car les angles sont des grandeurs sans dimensions} 
$$ = 1\Omega,\ CQFD $$
\end{center}
--- --- --- --- --- --- --- --- --- --- --- --- --- --- --- --- --- --- --- --- --- --- --- --- --- ---\\

\subsection*{2.2 Rappels th\'eoriques n\'ecessaires aux calculs}
$\bullet$Capacit\'e d'un circuit \`a la r\'esonance dans un circuit RLC s\'erie.
$$ \omega^{2} = \frac{1}{L \cdot C},\ \omega \ etant\ la\ vitesse\ angulaire\ a\ la\ resonance $$
\hspace*{0.2cm}Cf. page 6 de l'\'enonc\'e du labo.
$$ \Longleftrightarrow C = \frac{1}{L \cdot \omega^{2}} $$
$\bullet$L'intensit\'e maximale du courant (\`a la r\'esonance) dans un circuit RLC s\'erie.
$$ I_{eff\ max} = \frac{U_{eff}}{R_{tot}} $$ 
\hspace*{0.2cm}Cf. page 6 de l'\'enonc\'e du labo.\\ ~~\\
$\bullet$Potentiel aux bornes de L et aux bornes de C.
$$ U_{L} = Z_{L}\cdot I_{L},\ avec\ Z_{L} = \omega L,\ dans\ les\ reels $$ 
$$ U_{C} = Z_{C}\cdot I_{C},\ avec\ Z_{C} = \frac{1}{\omega C},\ dans\ les\ reels $$
\hspace*{0.2cm}Cf. page 3 de l'\'enonc\'e du labo.\\ ~~\\
$\bullet$Intensit\'e du courant aux bornes de L et aux bornes de C.
$$ \vert I_{L}\vert = \frac{\vert U_{L} \vert}{\omega L},\ dans\ les\ reels $$ 
$$ \vert I_{C}\vert = \vert U_{C} \vert \cdot \omega C,\ dans\ les\ reels $$
\hspace*{0.2cm}Cf. page 3 de l'\'enonc\'e du labo.\\ ~~\\
$\bullet$Capacit\'e d'un circuit \`a l' anti-r\'esonance dans un circuit RLC en parall\`ele.
$$ \omega^{2} = \frac{1}{L \cdot C},\ \omega \ etant\ la\ vitesse\ angulaire\ a\ la\ resonance $$
\hspace*{0.2cm}Cf. page 9 de l'\'enonc\'e du labo.
$$ \Longleftrightarrow C = \frac{1}{L \cdot \omega^{2}} $$
$\bullet$L'intensit\'e minimale du courant (\`a l' anti-r\'esonance) dans un circuit RLC en parall\`ele.
$$ I_{eff\ min} = \frac{U_{eff} \cdot R}{\omega L \cdot \sqrt{R^{2} + \omega^{2} L^{2}}} $$ 
\newpage

\section*{3 Dispositif exp\'erimental}

\subsection*{3.1 Appareils de mesure}

Voici quelques remarques sur les appareils de mesure utilis\'es: \\ ~~\\
\textbf{$\circledcirc$Amp\`erem\`etres \`a aiguille} \\ ~~\\
\hspace*{0,5cm}\ Nous avons utilis\'e deux niveaux de pr\'ecision sur ces appareils durant la manipulation.\ La r\'esistance interne de l'amp\`erem\`etre variait selon la pr\'ecision choisie. \\
\hspace*{0,59cm} Les pr\'ecisions suivantes repr\'esentent la graduation la plus \'elev\'ee pour la mesure. \\

\textbf{$\bullet$0,01A:} $Erreur\ absolue = \varepsilon_{A} = 2,5\% \cdot 0,01A = 0,00025A$ \\
\hspace*{1,82cm} $Resistance\ interne = R_{amp} = 9,85\Omega $ \\

\textbf{$\bullet$0,05A:} $Erreur\ absolue = \varepsilon_{A} = 2,5\% \cdot 0,05A = 0,00125A$ \\
\hspace*{1,82cm} $Resistance\ interne = R_{amp} = 3,6\Omega $


\subsection*{3.2 RLC s\'erie en tension alternative: r\'esonance}

Nous avons r\'ealis\'e le circuit repr\'esent\'e par le sch\'ema suivant lors de l'exp\'erience \\

\textbf{Sch\'ema de l'exp\'erience: }

\begin{figure}[h!]
\centering
\caption{Source: \'enonc\'e du labo}
\end{figure}

\pagebreak

\textbf{Conditions de travail: }

\begin{figure}[h!]
\centering
\caption{Source: \'enonc\'e du labo}
\end{figure}

\subsection*{3.3 RL \& C en parall\`ele: anti-r\'esonance}

Nous avons r\'ealis\'e le circuit repr\'esent\'e par le sch\'ema suivant lors de l'exp\'erience \\

\textbf{Sch\'ema de l'exp\'erience: }
\begin{figure}[h!]
\centering
\caption{Source: \'enonc\'e du labo}
\end{figure}

\textbf{Conditions de travail: }
\begin{figure}[h!]
\centering
\caption{Source: \'enonc\'e du labo}
\end{figure}

\section*{4 Prise des mesures \& r\'esultats}

\subsection*{4.1 RLC s\'erie en tension alternative: r\'esonance}

\hspace*{0,5cm} Comme dit dans l'illustration du point 3.2, Nous avons effectu\'e nos mesures en faisant varier la capacit\'e du condensateur entre 0,1 et 1$\mu F$ par pas de 0,1$\mu F$, sauf lorsque l'on approchait du point de r\'esonance, alors, on avan\c cait par pas de 0,01$\mu F$.  \\

\begin{tabular}{|c|c|c|c|}
\hline
\textbf{Capacit\'e} & \textbf{Intensit\'e du courant} & \textbf{Pr\'ecision amp\`erem\`etre} & \textbf{Erreur absolue} \\
\hline
0$\mu F$ & 0$A$ & --- & --- \\
\hline
0,1$\mu F$ & 0,0005$A$ & 0,01$A$ & 0,00025$A$ \\
\hline
0,2$\mu F$ & 0,0016$A$ & 0,01$A$ & 0,00025$A$ \\
\hline
0,3$\mu F$ & 0,0042$A$ & 0,01$A$ & 0,00025$A$ \\
\hline
0,4$\mu F$ & 0,0175$A$ & 0,05$A$ & 0,00125$A$ \\
\hline
0,41$\mu F$ & 0,0220$A$ & 0,05$A$ & 0,00125$A$\\
\hline
0,42$\mu F$ & 0,0295$A$ & 0,05$A$ & 0,00125$A$ \\
\hline
0,43$\mu F$ & 0,0385$A$ & 0,05$A$ & 0,00125$A$ \\
\hline
\textbf{0,44}$\mu F$ & \textbf{0,0440}$A$ & \textbf{0,05}$A$ & \textbf{0,00125}$A$ \\
\hline
0,45$\mu F$ & 0,0410$A$ & 0,05$A$ & 0,00125$A$ \\
\hline
0,46$\mu F$ & 0,0325$A$ & 0,05$A$ & 0,00125$A$ \\
\hline\
0,47$\mu F$ & 0,0260$A$ & 0,05$A$ & 0,00125$A$ \\
\hline\
0,48$\mu F$ & 0,0210$A$ & 0,05$A$ & 0,00125$A$ \\
\hline\
0,49$\mu F$ & 0,0180$A$ & 0,05$A$ & 0,00125$A$ \\
\hline\
0,5$\mu F$ & 0,0155$A$ & 0,05$A$ & 0,00125$A$ \\
\hline
0,6$\mu F$ & 0,0072$A$ & 0,01$A$ & 0,00025$A$ \\
\hline
0,7$\mu F$ & 0,0053$A$ & 0,01$A$ & 0,00025$A$ \\
\hline
0,8$\mu F$ & 0,0043$A$ & 0,01$A$ & 0,00025$A$ \\
\hline
0,9$\mu F$ & 0,0039$A$ & 0,01$A$ & 0,00025$A$ \\
\hline
1$\mu F$ & 0,0035$A$ & 0,01$A$ & 0,00025$A$ \\
\hline
\end{tabular}\\ ~~\\ ~~\\

\pagebreak

\subsection*{4.2 RL \& C en parall\`ele: anti-r\'esonance}

\hspace*{0,5cm} Comme dit dans l'illustration du point 3.3, Nous avons effectu\'e nos mesures en faisant varier la capacit\'e du condensateur entre 0,1 et 1$\mu F$ par pas de 0,1$\mu F$, sauf lorsque l'on approchait du point d'anti-r\'esonance, alors, on avan\c cait par pas de 0,01$\mu F$. \\

\begin{tabular}{|c|c|c|c|c|c|}
\hline
\textbf{Capacit\'e} & \textbf{$I$} & \textbf{$I_C$} & \textbf{$I_L$} & \textbf{Pr\'ecision amp\`erem\`etre} & \textbf{Erreur absolue} \\
\hline
0$\mu F$ & 2,25$A$ & 0$A$ & 2,25$A$ & 0,05$A$|0,01$A$|0,05$A$ & 0,00125$A$|0,00025$A$|0,00125$A$ \\
\hline
0,1$\mu F$ & 1,75$A$ & 0,52$A$ & 2,25$A$ & 0,05$A$|0,01$A$|0,05$A$ & 0,00125$A$|0,00025$A$|0,00125$A$ \\
\hline
0,2$\mu F$ & 2,19$A$ & 0,99$A$ & 2,25$A$ & 0,05$A$|0,01$A$|0,05$A$ & 0,00125$A$|0,00025$A$|0,00125$A$ \\
\hline
0,3$\mu F$ & 0,75$A$ & 1,50$A$ & 2,25$A$ & 0,01$A$|0,05$A$|0,05$A$ & 0,00025$A$|0,00125$A$|0,00125$A$ \\
\hline
0,4$\mu F$ & 0,24$A$ & 2,00$A$ & 2,25$A$ & 0,01$A$|0,05$A$|0,05$A$ & 0,00025$A$|0,00125$A$|0,00125$A$ \\
\hline
0,41$\mu F$ & 0,19$A$ & 2,00$A$ & 2,25$A$ & 0,01$A$|0,05$A$|0,05$A$ & 0,00025$A$|0,00125$A$|0,00125$A$ \\
\hline
0,42$\mu F$ & 0,15$A$ & 2,00$A$ & 2,25$A$ & 0,01$A$|0,05$A$|0,05$A$ & 0,00025$A$|0,00125$A$|0,00125$A$ \\
\hline
0,43$\mu F$ & 0,11$A$ & 2,00$A$ & 2,25$A$ & 0,01$A$|0,05$A$|0,05$A$ & 0,00025$A$|0,00125$A$|0,00125$A$ \\
\hline
\textbf{0,44$\mu F$} & \textbf{0,09$A$} &\textbf{ 2,05$A$} & \textbf{2,25$A$} & \textbf{0,01$A$|0,05$A$|0,05$A$} & \textbf{0,00025$A$|0,00125$A$|0,00125$A$} \\
\hline
0,45$\mu F$ & 0,10$A$ & 2,20$A$ & 2,25$A$ & 0,01$A$|0,05$A$|0,05$A$ & 0,00025$A$|0,00125$A$|0,00125$A$ \\
\hline
0,46$\mu F$ & 0,14$A$ & 2,20$A$ & 2,25$A$ & 0,01$A$|0,05$A$|0,05$A$ & 0,00025$A$|0,00125$A$|0,00125$A$ \\
\hline
0,47$\mu F$ & 0,17$A$ & 2,25$A$ & 2,25$A$ & 0,01$A$|0,05$A$|0,05$A$ & 0,00025$A$|0,00125$A$|0,00125$A$ \\
\hline
0,48$\mu F$ & 0,22$A$ & 2,25$A$ & 2,25$A$ & 0,01$A$|0,05$A$|0,05$A$ & 0,00025$A$|0,00125$A$|0,00125$A$ \\
\hline
0,49$\mu F$ & 0,26$A$ & 2,45$A$ & 2,25$A$ & 0,01$A$|0,05$A$|0,05$A$ & 0,00025$A$|0,00125$A$|0,00125$A$ \\
\hline
0,5$\mu F$ & 0,31$A$ & 2,55$A$ & 2,25$A$ & 0,01$A$|0,05$A$|0,05$A$ & 0,00025$A$|0,00125$A$|0,00125$A$ \\
\hline
0,6$\mu F$ & 0,83$A$ & 2,95$A$ & 2,25$A$ & 0,01$A$|0,05$A$|0,05$A$ & 0,00025$A$|0,00125$A$|0,00125$A$ \\
\hline
0,7$\mu F$ & 1,30$A$ & 3,55$A$ & 2,25$A$ & 0,05$A$|0,05$A$|0,05$A$ & 0,00125$A$|0,00125$A$|0,00125$A$ \\
\hline
0,8$\mu F$ & 1,85$A$ & 3,95$A$ & 2,25$A$ & 0,05$A$|0,05$A$|0,05$A$ & 0,00125$A$|0,00125$A$|0,00125$A$ \\
\hline
0,9$\mu F$ & 2,35$A$ & 4,50$A$ & 2,25$A$ & 0,05$A$|0,05$A$|0,05$A$ & 0,00125$A$|0,00125$A$|0,00125$A$ \\
\hline
1$\mu F$ & 2,90$A$ & 5,00$A$ & 2,25$A$ & 0,05$A$|0,05$A$|0,05$A$ & 0,00125$A$|0,00125$A$|0,00125$A$ \\
\hline


\end{tabular}\\ ~~\\ ~~\\


\pagebreak
\section*{5 Analyse des r\'esultats}

\subsection*{5.1 RLC s\'erie en tension alternative: r\'esonance}

Nous avons donc fait varier la capacit\'e du condensateur, comme d\'ecrit au point pr\'ec\'edent, et mesur\'e l'intensit\'e du courant gr\^ace \`a l'amp\`erem\`etre \`a aiguille (repr\'esent\'e sur le sch\'ema du point pr\'ec\'edent). \\

\textbf{Graphe de I en fonction de C: } \\ 
(N.B. l'unit\'e de l'axe des ordonn\'ees est le centi\`eme d'amp\`ere).
\begin{figure}[h]
\centering
\end{figure}

Nous avons donc que la capacit\'e \`a laquelle le circuit atteint le point de r\'esonance est $0,44\mu F$ et que l'intensit\'e du courant en ce point vaut $0,04400A \pm 0,00125A$. \\
L'erreur relative en ce point vaut: $$\frac{0,00125A}{0,04400A} = 0,02841$$

\pagebreak

\textbf{$\blacktriangleright$\'Enonc\'e:} Comparer la valeur de C pour laquele vous avez observ\'e \\
\hspace*{2,35cm}la r\'esonance avec la valeur th\'eorique attendue. \\
\hspace*{2,35cm}Observe-t-on une diff\'erence ? Si oui, expliquez. \\ 

\textbf{$\vartriangleright$R\'eponse:} ~~\\ 

$\circ$Nous savons que la capacit\'e du circuit \`a la r\'esonance vaut:
$$ C = \frac{1}{L \cdot \omega^{2}} $$ 

$\circ$Or, ici on a:
$$ L = 1,458H\ ;\ \omega = 2\pi f = 2\pi \cdot 200 = 1256,64m/s^{2} $$

$\circ$Donc, cel\`a nous donne:
$$ C = \frac{1}{1,458 \cdot (1256,64)^{2}} = 0,000000434F = 0,434\mu F $$

$\circ$La valeur observ\'ee exp\'erimentalement \'etait $0,44\mu F$\\
\hspace*{0.75cm}L'erreur de 6 milli\`eme de $\mu F$ est expliqu\'ee par le fait qu'il nous \'etait \\ 
\hspace*{0.75cm}demand\'e d'\^etre pr\'ecis au centi\`eme de $\mu F$ aux alentours du point de r\'esonance, \\
\hspace*{0.75cm}et non au milli\`eme de $\mu F$. \\
\hspace*{0.75cm}De plus, en jouant sur l'erreur absolue, on voit que notre r\'eponse est \\
\hspace*{0.75cm}correcte.
~~\\

--- --- --- --- --- --- --- --- --- --- --- --- --- --- --- --- --- --- --- --- --- --- --- --- \\

\textbf{$\blacktriangleright$\'Enonc\'e:} Comparer la valeur du courant obtenue \`a la r\'esonance avec la \\
\hspace*{2,35cm}valeur th\'eorique. Doit-on tenir compte de la r\'esistance interne \\
\hspace*{2,35cm}de l'amp\`erem\`etre dans le calcul ? \\ 

\textbf{$\vartriangleright$R\'eponse:} ~~\\

$\circ$Nous savons que l'intensit\'e du courant \`a la r\'esonance vaut :
$$ I_{eff\ res} = \frac{U_{eff}}{R_{res}} $$ 

$\circ$Or, nous savons qu'ici, $$U_{eff} = 3,5V\ ;\ R_{res} = R + R_{amp} = 75,78\Omega + 3,60\Omega = 79,38\Omega$$
\hspace*{0.75cm}En effet, on doit tenir compte de la r\'esistance interne de l'amp\`erem\`etre\\
\hspace*{0.75cm}\'etant donn\'e que c'est l'appareil de mesure. \\

\pagebreak

$\circ$Donc, cel\`a nous donne: $$ I_{eff\ res} = \frac{3,5V}{79,38\Omega} = 0,0440A $$
\hspace*{0.75cm}Nous trouvons donc la m\^eme valeur que celle obtenue de fa\c con exp\'erimentale \\

--- --- --- --- --- --- --- --- --- --- --- --- --- --- --- --- --- --- --- --- --- --- --- --- --- \\

\textbf{$\blacktriangleright$\'Enonc\'e:} Calculer les valeurs th\'eoriquement attendues pour $U_{L}$ et $U_{C}$ \\
\hspace*{2,35cm}\`a la r\'esonance et comparer aux valeurs mesur\'ees. \\

\textbf{$\vartriangleright$R\'eponse:} ~~\\

$\circ$Nous savons que:
$$ U_{L} = Z_{L}\cdot I_{L},\ avec\ Z_{L} = \omega L,\ dans\ les\ reels $$ 
$$ U_{C} = Z_{C}\cdot I_{C},\ avec\ Z_{C} = \frac{1}{\omega C},\ dans\ les\ reels $$
$$ L = 1,458H\ et\ \omega = 2\pi f = 2\pi \cdot 200Hz = 1256,637rad $$

$\circ$Comme nous somme \`a la r\'esonance;
$$ I_{L} = I_{C} = 0,0440A $$
$$ C = 0,44\mu F = 0,00000044F $$

$\circ$Donc, on a:
$$ Z_{L} = 1256,637rad \cdot 1,458H = 1832,177\Omega $$
$$ Z_{C} = \frac{1}{1256,637rad \cdot 0,00000044 F} = 1808,579\Omega $$

$\circ$Enfin, on trouve:
$$ U_{L} = 1832,177\Omega \cdot 0,04400A = 80,619V$$
$$ U_{C} = 1808,579\Omega \cdot 0,04400A = 79,578V$$


\newpage

\subsection*{5.2 RL \& C en parall\`ele: anti-r\'esonance}

Nous avons donc fait varier la capacit\'e du condensateur, comme d\'ecrit au point pr\'ec\'edent, et mesur\'e $I_{C}$, $I_{L}$ et $I$ gr\^ace aux amp\`ereme\`etres \`a aiguille (repr\'esent\'es sur le sch\'ema du point pr\'ec\'edent). \\

\textbf{Graphe des I en fonction de C:} \\ 
(N.B. l'unit\'e de l'axe des ordonn\'ees est le centi\`eme d'amp\`ere) 
\begin{figure}[h]
\centering
\end{figure}

\hspace*{0.5cm}Nous avons donc que la capacit\'e \`a laquelle le circuit atteint le point d'anti-r\'esonance est $0,44\mu F$ et qu'en ce point: 
$$I = 0,00090A \pm 0,00025A,\ erreur\ relative\ = \frac{0,00025}{0,00090} = 0,278$$
$$I_C = 0,02050A \pm 0,00125A,\ erreur\ relative\ = \frac{0,00125}{0,02050} = 0,061$$
$$I_L = 0,02250A \pm 0,00125A,\ erreur\ relative\ = \frac{0,00125}{0,02250} = 0,056$$

\pagebreak

\textbf{$\blacktriangleright$\'Enonc\'e:} Justifier l'allure des courbes \`a partir de la partie th\'eorique \\

\textbf{$\vartriangleright$R\'eponse:}  \\ 

$\circ$En ce qui concerne $I_{L}$, sa formule (cf. point 2.2) ne d\'epend en aucun cas\\ \hspace*{0.75cm}de C, c'est donc normal que sa valeur reste la m\^eme tandis que C varie.\\

$\circ$Pour $I_{C}$, sa formule (voir point 2.2) nous dit qu'elle est directement\\ \hspace*{0.75cm}proportionnelle \`a C, il est donc normal qu'elle cro\^it avec C. \\

$\circ$Enfin, $I_{res}$ d\'ecro\^it jusqu'\`a un minimum avant de cro\^itre de nouveau, comme\\
\hspace*{0.75cm}pr\'evu par le ph\'enom\`ene d'anti-r\'esonance. \\

--- --- --- --- --- --- --- --- --- --- --- --- --- --- --- --- --- --- --- --- --- --- --- \\

\textbf{$\blacktriangleright$\'Enonc\'e:} Calculer la valeur de C correspondant \`a l'anti-r\'esonance et \\
\hspace*{2,35cm}comparer \`a la valeur exp\'erimentale. \\

\textbf{$\vartriangleright$R\'eponse:} \\

$\circ$Nous savons que la capacit\'e du circuit \`a l'anti-r\'esonance vaut:
$$ C = \frac{1}{L \cdot \omega^{2}} $$ 

$\circ$Or, ici on a:
$$ L = 1,458H\ ;\ \omega = 2\pi f = 2\pi \cdot 200 = 1256,64m/s^{2} $$

$\circ$Donc, cel\`a nous donne:
$$ C = \frac{1}{1,458 \cdot (1256,64)^{2}} = 0,000000434F = 0,434\mu F $$

$\circ$La valeur observ\'ee exp\'erimentalement \'etait $0,44\mu F$\\
\hspace*{0.75cm}L'erreur de 6 milli\`eme de $\mu F$ est expliqu\'ee par le fait qu'il nous \'etait \\ 
\hspace*{0.75cm}demand\'e d'\^etre pr\'ecis au centi\`eme de $\mu F$ aux alentours du point de r\'esonance, \\
\hspace*{0.75cm}et non au milli\`eme de $\mu F$. \\
\hspace*{0.75cm}De plus, en jouant sur l'erreur absolue, on voit que notre r\'eponse est \\
\hspace*{0.75cm}correcte
~~\\

--- --- --- --- --- --- --- --- --- --- --- --- --- --- --- --- --- --- --- --- --- --- --- \\

\newpage

\textbf{$\blacktriangleright$\'Enonc\'e:} Calculer la valeur minimale attendue pour I et comparer \`a la \\
\hspace*{2,35cm}valeur exp\'erimentale. + En d\'eduire la valeur de la r\'esistance \\
\hspace*{2,35cm}\'equivalente du circuit antir\'esonant et comparer \`a la valeur \\
\hspace*{2,35cm}exp\'erimentale.\\

\textbf{$\vartriangleright$R\'eponse:} \\ ~~\\
$\circ$Nous savons que 
$$ I_{eff\ min} = \frac{U_{eff} \cdot R}{\omega L \cdot \sqrt{R^{2} + \omega^{2} L^{2}}} $$ 
$\circ$Or, nous avons ici
$$\omega = 1256,64m/s^{2}\ (voir\ calculs\ precedent),\ U_{eff} = 40V,\ R = 75,78\Omega,\ L = 1,458H$$
$\circ$Donc,
$$ I_{eff\ min} = \frac{40 \cdot 75,78}{1256,64 \cdot 1,458 \cdot \sqrt{75,78^{2} + 1256,64^{2} \cdot 1,458^{2}}}$$
$$  = \frac{3031,2}{1832,18 \cdot \sqrt{3362630,26}} = 0,00090A $$

Ce qui correspond parfaitement \`a la valeur ep\'erimentale.\\ ~~\\
$\circ$On peut en d\'eduire la r\'esistance effective du circuit par la formule suivante :
$$ R_{eff} = \frac{U_{eff}}{I_{eff}} $$
$$ \Longleftrightarrow R_{eff} = \frac{40}{0,00090} = 44444,44\Omega$$
\section*{6 Conclusion}
\hspace*{1cm}Nous avons exp\'eriment\'e le comportement de deux circuits RLC dispos\'es diff\'eremment: un en s\'erie, l'autre en parall\`ele. \\
\hspace*{1cm}Dans le premier, nous avons observ\'e le ph\'enom\`ene de r\'esonance, c\`ad l'existence d'une intensit\'e de courant maximale pour certaines valeurs. Dans ce cas, nous avons fait varier la capacit\'e du circuit pour atteindre la capacit\'e de r\'esoncance.\\
\hspace*{1cm}Dans le second, nous avons observ\'e le ph\'enom\`ene d'anti-r\'esonance, c\`ad l'existence d'une intensit\'e de courant minimale pour certaines valeurs. Dans ce cas, nous avons fait varier la capacit\'e du circuit pour atteindre la capacit\'e d' anti-r\'esoncance.\\

\end{document}
