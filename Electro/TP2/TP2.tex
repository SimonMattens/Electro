\documentclass{report}
\usepackage[pdftex]{graphicx}
\usepackage{sidecap}
\usepackage{fancyhdr}
\usepackage{lscape}
\usepackage[absolute]{textpos}
\usepackage{amssymb}
\usepackage{french}

\title{Rapport : Les circuits logiques}
\author{Mattens Simon, Dom Eduardo\\ BAB2 Sciences Informatiques}
\date{\today}


\begin{document}
\maketitle

\section*{1 Introduction}

\hspace*{1,5cm}Le but de cette manip est de comprendre et d'\^etre capable de construire des fonctions logiques \`a partir de fonctions logiques \'el\'ementaires. En ce basant sur ces fonctions logiques, nous avons construit des circuits \'electriques. Ces circuits \'etaient simples dans un premier temps, puis plus complexes et contenant des circuits int\'egr\'es logiques.\\

\section*{2 R\'esum\'e Th\'eorique}
\subsection*{2.1 Les fonctions logiques}
\hspace*{1,5cm} Une fonction logique est une logique dans laquelle on utilise que des variables dites binaires. Une variables binaire ne peut prendre que deux valeurs, not\'ees 1 et 0. Une fonction logique prend elle aussi des valeurs \'egales \`a 1 ou 0. On peut donner pour une fonction logique une table de v\'erit\'e qui reprend toute les valeurs de la fonction par rapport aux valeurs des variables.\\

On peut \'egalement \'ecrire une fonction logique ou une variable logique "A" par une fonction caract\'eristique not\'ee $f_{A}$(x) o\`u "x" est la variable r\'eelle valant soit 1 soit 0 et x$\in$A ou x$\in\overline{A}.$\\
\subsection*{2.2 Les fonctions logiques \'elementaires}
On retrouve 4 op\'erations logiques \'elementaires:
\begin{enumerate}
\item L'op\'eration NOT(non) qui est \'equivalent \`a la fonction $ \in\overline{A}$ . Elle vaut donc 0 quand x$\in$A et 1 quand x$\in\overline{A}$.\\


\newpage

\item L'op\'eration AND(et) correspond au produit logique A.B. Elle vaut 1 quand A$\cup$B est vrai , 0 sinon.\\
\item L'op\'eration OR(ou) correspond \`a la somme logique A+B. Elle vaut 1 quand A$\cap$B est vrai, 0 sinon.\\
\item L'operation XOR(ou exclusif) correspond \`a la somme directe A$\oplus$B. Elle vaut 1 quand (A$\cap\overline{B}$)$\cup$($\overline{A}\cap$B).\\ 
\end{enumerate}
On retrouve aussi des fonctions logiques telles que NAND (NON-ET) et NOR (NON-OU).
\newpage
\subsection*{2.3 Les circuits logiques}

Un circuit logique est un circuit \'electrique (ou \'electronique)qui repr\'esente une fonction logique. A l'aide d'un interupteur, on peut repr\'esent\'e la valeur 1 avec un interrupteur ferm\'e et la valeur 0 avec l'interrupteur ouvert. Le m\^eme genre de m\'echanisme existe avec des diodes.\\
On utilise des pictogrammes pour repr\'esenter des fonctions \'elementaires dans un circuit \'electrique.
\newpage
\section*{3 Dispositif exp\'erimental}
\subsection*{3.1 Outils}
\subsubsection*{3.1.1 Boitier}
\hspace*{1,2cm}Le boitier dispose des \'elements suivants:
\begin{itemize}
\item 4 interrupteurs à 2 directions, qui sont ouvert(0) et ferm\'e(1), chacun muni d'une LED qui indique sa direction.
\item 4 LEDs de contr\^ole permettant de d\'eterminer l'\'etat du circuit.
\item une plaquette de connexion.
\end{itemize}
\subsubsection*{3.1.2 Circuit int\'egr\'e SN7400}
\hspace*{1,5cm} Ce circuit comporte 4 circuits logiques simples identiques. Nous devrons d\'eterminer la fonction logique de ces 4 circuits durant la seconde exp\'erience.\\

\subsubsection*{3.1.3 Autres}
\hspace*{1,5cm} On dipsose aussi de r\'esistances, de fils de connexion ainsi que d'une source de tension r\'egl\'ee \`a 5V.\\
\newpage

\subsection*{3.2 Les circuits}
\hspace*{1,5cm} Nous avons eu besoin de faire les circuits suivants dans la premi\`ere exp\'erience :\\
\\
\newpage
\hspace*{1,5cm} Pour la seconde exp\'erience nous avons fait les circuits suivants: \\
\subsubsection*{3.2.1 Premier circuit}
\subsubsection*{3.2.2 Court-circuitage}
\subsubsection*{3.2.3 Les Montages} 
\newpage
\section*{4 Mesure et r\'esultat}
\subsection*{4.1 Premi\`ere exp\'erience}
\subsubsection*{4.1.1 Pr\'evision des r\'esultats}
\begin{enumerate}
\item Vu que l'interrupteur est en parall\`ele avec la LED, on peut pr\'edire que si celui-ci est ferm\'e alors le courant passera par cet interrupteur et pas par la LED (r\'esistance de la LED est sup\'erieure \`a 0 tandis que la r\'esistance interrupteur $\simeq$ 0). On soup\c conne donc un NON.
\item Vu qu'on retrouve 2 interrupteur et la LED en s\'erie tout les trois, les deux interrupteurs doivent \^etre ferm\'es pour que le courant passe. On pr\'evoit donc un ET.
\item Pour que le courant passe dans la LED, le courant doit pass\'e soit dans l'interrupteur A, soit dans le B. On pr\'evoit donc un OU.
\item Pour que le circuit soit ferm\'e, il faut les deux interrupteurs soient dans la m\^eme position. L'Op\'eration XOR correspondrait au cas inverse. On peut donc pr\'evoir un NON-XOR.
\item Vu que les interrupteurs sont en parall\`eles par rapport \`a la LED et qu'ils sont parall\`eles entre eux, la LED ne s'allumera que si le courant ne passe dans aucun des deux interrupteurs, c'est-\`a-dire si les deux interrupteurs sont ouverts. On pr\'evoit un NON-OU.
\item Vu que les deux interrupteurs sont en parallèles avec la LED et qu'ils sont en s\'erie entre eux, la LED ne s'allumera que si le courant passe dans la s\'erie d'interrupteur, c'est-\`a-dire si l'un des deux interrupteurs est ouvert. On pr\'evoit un NON-ET.
\end{enumerate}
\subsection*{4.1.2 Exp\'erimentation}
\begin{enumerate}
\item La LED est allum\'ee lorsque le bouton A est ouvert, \'eteinte si le bouton A est ferm\'e.
\item La LED est allum\'ee si les deux boutons sont ferm\'es , \'eteinte sinon.
\item La LED est \'eteinte si les deux boutons sont ouverts, allum\'ee sinon.
\item La LED est allum\'ee si les deux boutons sont sur la m\^eme position, \'eteinte sinon.
\item La LED n'est allum\'ee que si les deux boutons sont ouverts, \'eteinte sinon.
\item La LED n'est \'eteinte que si les deux boutons sont ferm\'es, allum\'ee sinon.

\end{enumerate}
\newpage
\subsection*{4.2 Seconde Expérience : Etude de circuist int\'egr\'es logiques}
\subsubsection*{4.2.1 Premier circuit}
\hspace*{1,5cm} La LED n'est \'eteinte que si les deux boutons sont ferm\'es, allum\'ee sinon.Ce qui donne la table suivante:\\
\begin{tabular}{|c|c|c|}
\hline
A & B & Exp \\
\hline
0&0&1\\
0&1&1\\
1&0&1\\
1&1&0\\
\hline
\end{tabular}\\
\subsubsection*{4.2.2 Court-circuitage les entr\'ees}
\hspace*{1,5cm} Si on court-circuites les deux entr\'ees, la LED on obtient la table
suivante:\\
\begin{tabular}{|c|c|}
\hline
A & Exp \\
\hline
0&1\\
1&0\\
\hline
\end{tabular}\\
\subsubsection*{4.2.3 Les montages}
\begin{enumerate}
\item La LED est allum\'ee si les deux boutons sont ferm\'es , \'eteinte sinon.
\item La LED est \'eteinte si les deux boutons sont ouverts, allum\'ee sinon.
\item La LED est allum\'ee si les deux boutons sont dans des \'etats inverses(l'un ouvert, l'autre ferm\'e), \'eteinte sinon.
\item La LED est allum\'ee si les deux boutons sont ouverts, \'eteinte sinon.
\end{enumerate}
\newpage
\section*{5 Analyse et r\'esultat}
\subsection*{5.1 Pr\'eliminaires}
\subsubsection*{5.1.1 Prouver A$\oplus$B = A.$\overline{B}$ + $\overline{A}$.B}
\hspace*{1,5cm} Afin de prouver que A$\oplus$B = A.$\overline{B}$ + $\overline{A}$.B est une tautologie, construisons la table de v\'erit\'e de l'expression.\\
\begin{tabular}{|c|c|c|c|c|c|c|c|}
\hline
 & & & & & & & \\ 
A & B & $\overline{A}$ & $\overline{B}$ & A.$\overline{B}$ & $\overline{A}$.B & A.$\overline{B}$ + $\overline{A}$.B & A$\oplus$B \\
\hline
0&0&1&1&0&0&0&0\\
0&1&1&0&0&1&1&1\\
1&0&0&1&1&0&1&1\\
1&1&0&0&0&0&0&0\\
\hline
\end{tabular}\\
\\
\hspace*{1,5cm} Nous voyons via cette table de v\'erit\'e que A.$\overline{B}$ + $\overline{A}$.B = A$\oplus$B.\\
\subsubsection*{5.1.2 Table NAND et prouver $\overline{A.B}$=$\overline{A}+\overline{B}$}
Table NAND: \\
\begin{tabular}{|c|c|c|c|}
\hline
  & & & \\
 A & B & A.B & $\overline{A.B}$\\
\hline
 0 & 0 & 0 & 1\\
 0 & 1 & 0 & 1\\
 1 & 0 & 0 & 1\\
 1 & 1 & 1 & 0 \\
 \hline
\end{tabular} \\

\hspace*{1,5cm}Afin de prouver que $\overline{A.B}$=$\overline{A}+\overline{B}$ est une tautologie ,construisons la table de v\'erit\'e de l'expression.\\

\begin{tabular}{|c|c|c|c|c|c|c|}
\hline
 & & & & & & \\
 A & B & $\overline{A}$ & $\overline{B}$ & $\overline{A}+\overline{B}$ & A.B & $\overline{A.B}$\\
 \hline
 0&0&1&1&1&0&1\\
 0&1&1&0&1&0&1\\
 1&0&0&1&1&0&1\\
 1&1&0&0&0&1&0\\
 \hline
\end{tabular}\\
\hspace*{1,5cm} Nous avons prouv\'e via cette table que l'expression $\overline{A.B}$=$\overline{A}+\overline{B}$ est une tautologie.\\
\newpage
\hspace*{1,5cm} Diagramme de Venn de l'expression:\\
\subsubsection*{5.1.3 Table NOR et prouver $\overline{A+B}$=$\overline{A}.\overline{B}$}

\hspace{1,5cm}Table NOR: \\

\begin{tabular}{|c|c|c|c|}
\hline
 & & & \\
 A & B & A+B & $\overline{A+B}$\\
 \hline
 0 & 0 & 0 & 1\\
 0 & 1 & 1 & 0\\
 1 & 0 & 1 & 0\\
 1 & 1 & 1 & 0 \\
 \hline
\end{tabular}\\

\hspace*{1,5cm}Afin de prouver que $\overline{A+B}$=$\overline{A}.\overline{B}$ est une tautologie ,construisons la table de v\'erit\'e de l'expression.\\

\begin{tabular}{|c|c|c|c|c|c|c|}
\hline
 & & & & & & \\ 
A & B & $\overline{A}$ & $\overline{B}$ & $\overline{A}$.$\overline{B}$ & A+B & $\overline{A+B}$ \\
\hline
0&0&1&1&1&0&1\\
0&1&1&0&0&1&0\\
1&0&0&1&0&1&0\\
1&1&0&0&0&1&0\\
\hline
\end{tabular}\\

\hspace*{1,5cm} Nous avons prouv\'e via cette table que l'expression $\overline{A+B}$=$\overline{A}.\overline{B}$ est une tautologie.\\
\newpage
\hspace*{1,5cm} Diagramme de Venn de l'expression:\\
\newpage
\subsection*{5.2 Premi\`ere exp\'erience}
\begin{enumerate}
\item On obtient les r\'esultats suivant:\\
\\
\begin{tabular}{|c|c|}
\hline
A & Exp \\
\hline
0&1\\
1&0\\
\hline
\end{tabular}\\
\hspace*{1,2cm} Ce qui correspond \`a l'Op\'eration NON.
\item On obtient les résultats suivants:\\
\\
\begin{tabular}{|c|c|c|}
\hline
A & B & Exp \\
\hline
0&0&0\\
0&1&0\\
1&0&0\\
1&1&1\\
\hline
\end{tabular}\\
\\
\hspace*{1,2cm} Ce qui correspond \`a l'Op\'eration ET.
\item On obtient les résultats suivants:\\
\\
\begin{tabular}{|c|c|c|}
\hline
A & B & Exp \\
\hline
0&0&0\\
0&1&1\\
1&0&1\\
1&1&1\\
\hline
\end{tabular}\\
\\
\hspace*{1,2cm} Ce qui correspond \`a l'Op\'eration OU.
\item On obtient les résultats suivants:\\
\\
\begin{tabular}{|c|c|c|}
\hline
A & B & Exp \\
\hline
0&0&1\\
0&1&0\\
1&0&0\\
1&1&1\\
\hline
\end{tabular}\\
\hspace*{1,2cm} Ce qui correspond \`a l'Op\'eration NON XOR.
\item On obtient les résultats suivants:\\
\\
\begin{tabular}{|c|c|c|}
\hline
A & B & Exp \\
\hline
0&0&1\\
0&1&0\\
1&0&0\\
1&1&0\\
\hline
\end{tabular}\\
\\
\hspace*{1,2cm} Ce qui correspond \`a l'Op\'eration NON-OU.
\newpage
\item On obtient les résultats suivants:\\
\\
\begin{tabular}{|c|c|c|}
\hline
A & B & Exp \\
\hline
0&0&1\\
0&1&1\\
1&0&1\\
1&1&0\\
\hline
\end{tabular}\\
\\
\hspace*{1,2cm} Ce qui correspond \`a l'Op\'eration NON-ET.
\end{enumerate}
\subsection*{5.2 Deuxi\`eme exp\'erience: circuits int\'egr\'es logiques}
\subsubsection*{5.2.1 Premier circuit}
\hspace*{1,5cm} Les \'etats correspondent au tableau de v\'erit\'e du sixi\`eme circuit de la premi\'ere exp\'erience. Nous avons donc un NON-ET.\\
\hspace*{2,0cm} \textbf{Nous avons donc 4 NON-ET dans le circuit int\'egr\'e.}
\subsubsection*{5.2.2 Court-circuitage des entr\'ees}
\hspace*{1,5cm} Soit A la premi\`ere entr\'ee et B la deuxième entr\'ee Lors du court-circuitage , obtient un NON. En effet, vu que nous avons un circuits NON-ET avec A=B, nous avons:\\
\\
\begin{tabular}{|c|c|c|}
\hline
A& B & Exp \\
\hline
0&0&1\\
1&1&0\\
\hline
\end{tabular}\\
\\
\hspace*{1,5cm} Ce qui correspond \`a un NON.
\subsubsection*{5.2.3 Les Montages}
\begin{enumerate}
\item Lors du premier montages on obtient:\\
\\
\begin{tabular}{|c|c|c|}
\hline
A & B & Exp \\
\hline
0&0&0\\
0&1&0\\
1&0&0\\
1&1&1\\
\hline
\end{tabular}\\
\\
Ce qui correspond \`a un ET. En effet, à la sortie 3, nous avons un NON-ET(voir 5.2.1) suivit d'un ET (voir 5.2.2).\\
\newpage
\item Lors du Deuxi\`eme montage, on obtient:\\
\begin{tabular}{|c|c|c|}
\hline
A & B & Exp \\
\hline
0&0&0\\
0&1&1\\
1&0&1\\
1&1&1\\
\hline
\end{tabular}\\
Ce qui correspond \`a un OU. En effet, \`a la sortie 3 nous avons $\overline{A}$ et \`a la sortie 6 nous avons $\overline{B}$. D\`es lors , lorsque A=B=0, on a $\overline{A}$=$\overline{B}$=1 donc le circuit vaut 1 (voir 5.2.1).\\
\item Lors du Troisi\`eme montage on obtient:\\
\\
\begin{tabular}{|c|c|c|}
\hline
A & B & Exp \\
\hline
0&0&0\\
0&1&1\\
1&0&1\\
1&1&0\\
\hline
\end{tabular}\\
\\
Ce qui correspond \`a un XOR. Afin de comprendre cel\`a, construisons la table des \'etats des entr\'ees et sorties du circuit:\\
\\
\begin{tabular}{|c|c|c|c|c|c|}
\hline
a=1=4 & B=2=10 & 3=5=9 & 6=12 & 8=13 & 11=EXP \\
\hline
0&0&1&1&1&0\\
0&1&1&1&0&1\\
1&0&1&0&1&1\\
1&1&0&1&1&0\\
\hline
\end{tabular} 
\item Lors du dernier montage, on obtient :\\
\\
\begin{tabular}{|c|c|c|}
\hline
A & B & Exp \\
\hline
0&0&1\\
0&1&0\\
1&0&0\\
1&1&1\\
\hline
\end{tabular}\\
\\
Ce qui correspond \`a un NON-OU. En effet, \`a la sortie 8, nous avons un OU (voir deuxi\`eme montage). Suivit d'un court-circuit, nous avons donc un NON-OU.\\
\end{enumerate}
\newpage
\section*{6 Conclusion}
\hspace*{1,5cm} Nous avons ,dans ce laboratoire, construit de nombreux circuits correspondant aux logiques \'el\'ementaire. Nous avons donc prouv\'e que chaque fonction logique pouvait \^etre construit sous la forme d'un circuit logique gr\^ace \`a des interrupteurs.\\
\hspace*{1,5cm} Nous avons aussi constat\'e que l'on pouvait construire plusieurs fonctions logiques \`a partir d'un circuit int\'egr\'e compos\'e uniquement de portes NAND. En particulier, la th\'eorie nous dit que nous pouvons construire n'importe quelle fonction logique \`a partir de portes NAND. 
\end{document}
\documentclass{report}
\usepackage[pdftex]{graphicx}
\usepackage{sidecap}
\usepackage{fancyhdr}
\usepackage{lscape}
\usepackage[absolute]{textpos}
\usepackage{amssymb}
\title{Compte-rendu de la Manip 5: Les circuits logiques}
\author{Van Herzeele Maxime; Rouneau Anthony; Potie Nicolas \\ BA2 Info}
\date{laboratoire du 22 avril 2014}

\pagestyle{fancy}
\lhead{Groupe 1: Van Herzeele M. ; Rouneau A. ; Potie N.}
\rhead{BA2 Info}
\cfoot{\thepage}
\begin{document}
\maketitle

\section*{1 Introduction}

\hspace*{1,5cm}Le but de cette manip est de comprendre et d'\^etre capable de construire des fonctions logiques \`a partir de fonctions logiques \'el\'ementaires. En ce basant sur ces fonctions logiques, nous avons construit des circuits \'electriques. Ces circuits \'etaient simples dans un premier temps, puis plus complexes et contenant des circuits int\'egr\'es logiques.\\

\section*{2 R\'esum\'e Th\'eorique}
\subsection*{2.1 Les fonctions logiques}
\hspace*{1,5cm} Une fonction logique est une logique dans laquelle on utilise que des variables dites binaires. Une variables binaire ne peut prendre que deux valeurs, not\'ees 1 et 0. Une fonction logique prend elle aussi des valeurs \'egales \`a 1 ou 0. On peut donner pour une fonction logique une table de v\'erit\'e qui reprend toute les valeurs de la fonction par rapport aux valeurs des variables.\\

On peut \'egalement \'ecrire une fonction logique ou une variable logique "A" par une fonction caract\'eristique not\'ee $f_{A}$(x) o\`u "x" est la variable r\'eelle valant soit 1 soit 0 et x$\in$A ou x$\in\overline{A}.$\\
\subsection*{2.2 Les fonctions logiques \'elementaires}
On retrouve 4 op\'erations logiques \'elementaires:
\begin{enumerate}
\item L'op\'eration NOT(non) qui est \'equivalent \`a la fonction $ \in\overline{A}$ . Elle vaut donc 0 quand x$\in$A et 1 quand x$\in\overline{A}$.\\


\newpage

\item L'op\'eration AND(et) correspond au produit logique A.B. Elle vaut 1 quand A$\cup$B est vrai , 0 sinon.\\
\item L'op\'eration OR(ou) correspond \`a la somme logique A+B. Elle vaut 1 quand A$\cap$B est vrai, 0 sinon.\\
\item L'operation XOR(ou exclusif) correspond \`a la somme directe A$\oplus$B. Elle vaut 1 quand (A$\cap\overline{B}$)$\cup$($\overline{A}\cap$B).\\ 
\end{enumerate}
On retrouve aussi des fonctions logiques telles que NAND (NON-ET) et NOR (NON-OU).
\newpage
\subsection*{2.3 Les circuits logiques}

Un circuit logique est un circuit \'electrique (ou \'electronique)qui repr\'esente une fonction logique. A l'aide d'un interupteur, on peut repr\'esent\'e la valeur 1 avec un interrupteur ferm\'e et la valeur 0 avec l'interrupteur ouvert. Le m\^eme genre de m\'echanisme existe avec des diodes.\\
On utilise des pictogrammes pour repr\'esenter des fonctions \'elementaires dans un circuit \'electrique.
\newpage
\section*{3 Dispositif exp\'erimental}
\subsection*{3.1 Outils}
\subsubsection*{3.1.1 Boitier}
\hspace*{1,2cm}Le boitier dispose des \'elements suivants:
\begin{itemize}
\item 4 interrupteurs à 2 directions, qui sont ouvert(0) et ferm\'e(1), chacun muni d'une LED qui indique sa direction.
\item 4 LEDs de contr\^ole permettant de d\'eterminer l'\'etat du circuit.
\item une plaquette de connexion.
\end{itemize}
\subsubsection*{3.1.2 Circuit int\'egr\'e SN7400}
\hspace*{1,5cm} Ce circuit comporte 4 circuits logiques simples identiques. Nous devrons d\'eterminer la fonction logique de ces 4 circuits durant la seconde exp\'erience.\\

\subsubsection*{3.1.3 Autres}
\hspace*{1,5cm} On dipsose aussi de r\'esistances, de fils de connexion ainsi que d'une source de tension r\'egl\'ee \`a 5V.\\
\newpage

\subsection*{3.2 Les circuits}
\hspace*{1,5cm} Nous avons eu besoin de faire les circuits suivants dans la premi\`ere exp\'erience :\\
\\
\newpage
\hspace*{1,5cm} Pour la seconde exp\'erience nous avons fait les circuits suivants: \\
\subsubsection*{3.2.1 Premier circuit}
\subsubsection*{3.2.2 Court-circuitage}
\subsubsection*{3.2.3 Les Montages} 
\newpage
\section*{4 Mesure et r\'esultat}
\subsection*{4.1 Premi\`ere exp\'erience}
\subsubsection*{4.1.1 Pr\'evision des r\'esultats}
\begin{enumerate}
\item Vu que l'interrupteur est en parall\`ele avec la LED, on peut pr\'edire que si celui-ci est ferm\'e alors le courant passera par cet interrupteur et pas par la LED (r\'esistance de la LED est sup\'erieure \`a 0 tandis que la r\'esistance interrupteur $\simeq$ 0). On soup\c conne donc un NON.
\item Vu qu'on retrouve 2 interrupteur et la LED en s\'erie tout les trois, les deux interrupteurs doivent \^etre ferm\'es pour que le courant passe. On pr\'evoit donc un ET.
\item Pour que le courant passe dans la LED, le courant doit pass\'e soit dans l'interrupteur A, soit dans le B. On pr\'evoit donc un OU.
\item Pour que le circuit soit ferm\'e, il faut les deux interrupteurs soient dans la m\^eme position. L'Op\'eration XOR correspondrait au cas inverse. On peut donc pr\'evoir un NON-XOR.
\item Vu que les interrupteurs sont en parall\`eles par rapport \`a la LED et qu'ils sont parall\`eles entre eux, la LED ne s'allumera que si le courant ne passe dans aucun des deux interrupteurs, c'est-\`a-dire si les deux interrupteurs sont ouverts. On pr\'evoit un NON-OU.
\item Vu que les deux interrupteurs sont en parallèles avec la LED et qu'ils sont en s\'erie entre eux, la LED ne s'allumera que si le courant passe dans la s\'erie d'interrupteur, c'est-\`a-dire si l'un des deux interrupteurs est ouvert. On pr\'evoit un NON-ET.
\end{enumerate}
\subsection*{4.1.2 Exp\'erimentation}
\begin{enumerate}
\item La LED est allum\'ee lorsque le bouton A est ouvert, \'eteinte si le bouton A est ferm\'e.
\item La LED est allum\'ee si les deux boutons sont ferm\'es , \'eteinte sinon.
\item La LED est \'eteinte si les deux boutons sont ouverts, allum\'ee sinon.
\item La LED est allum\'ee si les deux boutons sont sur la m\^eme position, \'eteinte sinon.
\item La LED n'est allum\'ee que si les deux boutons sont ouverts, \'eteinte sinon.
\item La LED n'est \'eteinte que si les deux boutons sont ferm\'es, allum\'ee sinon.

\end{enumerate}
\newpage
\subsection*{4.2 Seconde Expérience : Etude de circuist int\'egr\'es logiques}
\subsubsection*{4.2.1 Premier circuit}
\hspace*{1,5cm} La LED n'est \'eteinte que si les deux boutons sont ferm\'es, allum\'ee sinon.Ce qui donne la table suivante:\\
\begin{tabular}{|c|c|c|}
\hline
A & B & Exp \\
\hline
0&0&1\\
0&1&1\\
1&0&1\\
1&1&0\\
\hline
\end{tabular}\\
\subsubsection*{4.2.2 Court-circuitage les entr\'ees}
\hspace*{1,5cm} Si on court-circuites les deux entr\'ees, la LED on obtient la table
suivante:\\
\begin{tabular}{|c|c|}
\hline
A & Exp \\
\hline
0&1\\
1&0\\
\hline
\end{tabular}\\
\subsubsection*{4.2.3 Les montages}
\begin{enumerate}
\item La LED est allum\'ee si les deux boutons sont ferm\'es , \'eteinte sinon.
\item La LED est \'eteinte si les deux boutons sont ouverts, allum\'ee sinon.
\item La LED est allum\'ee si les deux boutons sont dans des \'etats inverses(l'un ouvert, l'autre ferm\'e), \'eteinte sinon.
\item La LED est allum\'ee si les deux boutons sont ouverts, \'eteinte sinon.
\end{enumerate}
\newpage
\section*{5 Analyse et r\'esultat}
\subsection*{5.1 Pr\'eliminaires}
\subsubsection*{5.1.1 Prouver A$\oplus$B = A.$\overline{B}$ + $\overline{A}$.B}
\hspace*{1,5cm} Afin de prouver que A$\oplus$B = A.$\overline{B}$ + $\overline{A}$.B est une tautologie, construisons la table de v\'erit\'e de l'expression.\\
\begin{tabular}{|c|c|c|c|c|c|c|c|}
\hline
 & & & & & & & \\ 
A & B & $\overline{A}$ & $\overline{B}$ & A.$\overline{B}$ & $\overline{A}$.B & A.$\overline{B}$ + $\overline{A}$.B & A$\oplus$B \\
\hline
0&0&1&1&0&0&0&0\\
0&1&1&0&0&1&1&1\\
1&0&0&1&1&0&1&1\\
1&1&0&0&0&0&0&0\\
\hline
\end{tabular}\\
\\
\hspace*{1,5cm} Nous voyons via cette table de v\'erit\'e que A.$\overline{B}$ + $\overline{A}$.B = A$\oplus$B.\\
\subsubsection*{5.1.2 Table NAND et prouver $\overline{A.B}$=$\overline{A}+\overline{B}$}
Table NAND: \\
\begin{tabular}{|c|c|c|c|}
\hline
  & & & \\
 A & B & A.B & $\overline{A.B}$\\
\hline
 0 & 0 & 0 & 1\\
 0 & 1 & 0 & 1\\
 1 & 0 & 0 & 1\\
 1 & 1 & 1 & 0 \\
 \hline
\end{tabular} \\

\hspace*{1,5cm}Afin de prouver que $\overline{A.B}$=$\overline{A}+\overline{B}$ est une tautologie ,construisons la table de v\'erit\'e de l'expression.\\

\begin{tabular}{|c|c|c|c|c|c|c|}
\hline
 & & & & & & \\
 A & B & $\overline{A}$ & $\overline{B}$ & $\overline{A}+\overline{B}$ & A.B & $\overline{A.B}$\\
 \hline
 0&0&1&1&1&0&1\\
 0&1&1&0&1&0&1\\
 1&0&0&1&1&0&1\\
 1&1&0&0&0&1&0\\
 \hline
\end{tabular}\\
\hspace*{1,5cm} Nous avons prouv\'e via cette table que l'expression $\overline{A.B}$=$\overline{A}+\overline{B}$ est une tautologie.\\
\newpage
\hspace*{1,5cm} Diagramme de Venn de l'expression:\\
\subsubsection*{5.1.3 Table NOR et prouver $\overline{A+B}$=$\overline{A}.\overline{B}$}

\hspace{1,5cm}Table NOR: \\

\begin{tabular}{|c|c|c|c|}
\hline
 & & & \\
 A & B & A+B & $\overline{A+B}$\\
 \hline
 0 & 0 & 0 & 1\\
 0 & 1 & 1 & 0\\
 1 & 0 & 1 & 0\\
 1 & 1 & 1 & 0 \\
 \hline
\end{tabular}\\

\hspace*{1,5cm}Afin de prouver que $\overline{A+B}$=$\overline{A}.\overline{B}$ est une tautologie ,construisons la table de v\'erit\'e de l'expression.\\

\begin{tabular}{|c|c|c|c|c|c|c|}
\hline
 & & & & & & \\ 
A & B & $\overline{A}$ & $\overline{B}$ & $\overline{A}$.$\overline{B}$ & A+B & $\overline{A+B}$ \\
\hline
0&0&1&1&1&0&1\\
0&1&1&0&0&1&0\\
1&0&0&1&0&1&0\\
1&1&0&0&0&1&0\\
\hline
\end{tabular}\\

\hspace*{1,5cm} Nous avons prouv\'e via cette table que l'expression $\overline{A+B}$=$\overline{A}.\overline{B}$ est une tautologie.\\
\newpage
\hspace*{1,5cm} Diagramme de Venn de l'expression:\\
\newpage
\subsection*{5.2 Premi\`ere exp\'erience}
\begin{enumerate}
\item On obtient les r\'esultats suivant:\\
\\
\begin{tabular}{|c|c|}
\hline
A & Exp \\
\hline
0&1\\
1&0\\
\hline
\end{tabular}\\
\hspace*{1,2cm} Ce qui correspond \`a l'Op\'eration NON.
\item On obtient les résultats suivants:\\
\\
\begin{tabular}{|c|c|c|}
\hline
A & B & Exp \\
\hline
0&0&0\\
0&1&0\\
1&0&0\\
1&1&1\\
\hline
\end{tabular}\\
\\
\hspace*{1,2cm} Ce qui correspond \`a l'Op\'eration ET.
\item On obtient les résultats suivants:\\
\\
\begin{tabular}{|c|c|c|}
\hline
A & B & Exp \\
\hline
0&0&0\\
0&1&1\\
1&0&1\\
1&1&1\\
\hline
\end{tabular}\\
\\
\hspace*{1,2cm} Ce qui correspond \`a l'Op\'eration OU.
\item On obtient les résultats suivants:\\
\\
\begin{tabular}{|c|c|c|}
\hline
A & B & Exp \\
\hline
0&0&1\\
0&1&0\\
1&0&0\\
1&1&1\\
\hline
\end{tabular}\\
\hspace*{1,2cm} Ce qui correspond \`a l'Op\'eration NON XOR.
\item On obtient les résultats suivants:\\
\\
\begin{tabular}{|c|c|c|}
\hline
A & B & Exp \\
\hline
0&0&1\\
0&1&0\\
1&0&0\\
1&1&0\\
\hline
\end{tabular}\\
\\
\hspace*{1,2cm} Ce qui correspond \`a l'Op\'eration NON-OU.
\newpage
\item On obtient les résultats suivants:\\
\\
\begin{tabular}{|c|c|c|}
\hline
A & B & Exp \\
\hline
0&0&1\\
0&1&1\\
1&0&1\\
1&1&0\\
\hline
\end{tabular}\\
\\
\hspace*{1,2cm} Ce qui correspond \`a l'Op\'eration NON-ET.
\end{enumerate}
\subsection*{5.2 Deuxi\`eme exp\'erience: circuits int\'egr\'es logiques}
\subsubsection*{5.2.1 Premier circuit}
\hspace*{1,5cm} Les \'etats correspondent au tableau de v\'erit\'e du sixi\`eme circuit de la premi\'ere exp\'erience. Nous avons donc un NON-ET.\\
\hspace*{2,0cm} \textbf{Nous avons donc 4 NON-ET dans le circuit int\'egr\'e.}
\subsubsection*{5.2.2 Court-circuitage des entr\'ees}
\hspace*{1,5cm} Soit A la premi\`ere entr\'ee et B la deuxième entr\'ee Lors du court-circuitage , obtient un NON. En effet, vu que nous avons un circuits NON-ET avec A=B, nous avons:\\
\\
\begin{tabular}{|c|c|c|}
\hline
A& B & Exp \\
\hline
0&0&1\\
1&1&0\\
\hline
\end{tabular}\\
\\
\hspace*{1,5cm} Ce qui correspond \`a un NON.
\subsubsection*{5.2.3 Les Montages}
\begin{enumerate}
\item Lors du premier montages on obtient:\\
\\
\begin{tabular}{|c|c|c|}
\hline
A & B & Exp \\
\hline
0&0&0\\
0&1&0\\
1&0&0\\
1&1&1\\
\hline
\end{tabular}\\
\\
Ce qui correspond \`a un ET. En effet, à la sortie 3, nous avons un NON-ET(voir 5.2.1) suivit d'un ET (voir 5.2.2).\\
\newpage
\item Lors du Deuxi\`eme montage, on obtient:\\
\begin{tabular}{|c|c|c|}
\hline
A & B & Exp \\
\hline
0&0&0\\
0&1&1\\
1&0&1\\
1&1&1\\
\hline
\end{tabular}\\
Ce qui correspond \`a un OU. En effet, \`a la sortie 3 nous avons $\overline{A}$ et \`a la sortie 6 nous avons $\overline{B}$. D\`es lors , lorsque A=B=0, on a $\overline{A}$=$\overline{B}$=1 donc le circuit vaut 1 (voir 5.2.1).\\
\item Lors du Troisi\`eme montage on obtient:\\
\\
\begin{tabular}{|c|c|c|}
\hline
A & B & Exp \\
\hline
0&0&0\\
0&1&1\\
1&0&1\\
1&1&0\\
\hline
\end{tabular}\\
\\
Ce qui correspond \`a un XOR. Afin de comprendre cel\`a, construisons la table des \'etats des entr\'ees et sorties du circuit:\\
\\
\begin{tabular}{|c|c|c|c|c|c|}
\hline
a=1=4 & B=2=10 & 3=5=9 & 6=12 & 8=13 & 11=EXP \\
\hline
0&0&1&1&1&0\\
0&1&1&1&0&1\\
1&0&1&0&1&1\\
1&1&0&1&1&0\\
\hline
\end{tabular} 
\item Lors du dernier montage, on obtient :\\
\\
\begin{tabular}{|c|c|c|}
\hline
A & B & Exp \\
\hline
0&0&1\\
0&1&0\\
1&0&0\\
1&1&1\\
\hline
\end{tabular}\\
\\
Ce qui correspond \`a un NON-OU. En effet, \`a la sortie 8, nous avons un OU (voir deuxi\`eme montage). Suivit d'un court-circuit, nous avons donc un NON-OU.\\
\end{enumerate}
\newpage
\section*{6 Conclusion}
\hspace*{1,5cm} Nous avons ,dans ce laboratoire, construit de nombreux circuits correspondant aux logiques \'el\'ementaire. Nous avons donc prouv\'e que chaque fonction logique pouvait \^etre construit sous la forme d'un circuit logique gr\^ace \`a des interrupteurs.\\
\hspace*{1,5cm} Nous avons aussi constat\'e que l'on pouvait construire plusieurs fonctions logiques \`a partir d'un circuit int\'egr\'e compos\'e uniquement de portes NAND. En particulier, la th\'eorie nous dit que nous pouvons construire n'importe quelle fonction logique \`a partir de portes NAND. 
\end{document}

